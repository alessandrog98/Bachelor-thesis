\chapter{Conclusioni}
\label{chap:Conclusion}
I test effettuati hanno permesso di determinare l'efficacia ancora attuale della \textbf{Return Oriented Programming}. Essa rappresenta tuttora una minaccia per i sistemi informatici 
e se sottovalutata potrebbe arrecare ad essi ingenti danni.\\
Spesso una apparentemente innocua disattenzione da parte del programmatore rappresenta la principale causa di rottura dell'integrità di un'applicazione, 
costituendo di fatto un possibile appoggio per un attacco da terzi.\\Tale tecnica è un potente strumento poiché in grado di adattarsi a molteplici sistemi e differenti situazioni che potrebbero essere riscontrate. 
Per la suddetta ragione molte rilevanti aziende nel campo informatico, come ad esempio \textbf{Intel} o \textbf{AMD}, nel corso degli anni hanno proposto diverse soluzioni volte ad evitare attacchi realizzati attraverso la \textbf{ROP}.\\
Per poter proteggere al meglio un'applicazione non solo diviene essenziale implementare tali strategie difensive, ma sarà necessario evitare in principio di introdurre all'interno di esse delle vulnerabilità.\\
Al giorno d'oggi esistono un elevato numero di tipologie differenti di cyber attacchi con effetti più o meno gravi sul sistema di destinazione, dunque diventa sempre più complesso lo sviluppo di difese efficaci 
che precludano di eseguire un'offensiva in qualsiasi caso. Nonostante la presenza di efficaci meccanismi difensivi, quali la \textbf{W$\oplus$X} oppure la \textbf{ASLR}, con l'esistenza di tecniche come quella
analizzata nel seguente elaborato, alcune delle zone di memoria di maggior importanza del sistema saranno comunque vulnerabili ad eventuali attacchi.\\
Un altro importante fattore da non sottovalutare è la sempre più crescente presenza di strumenti che rendono relativamente semplice ed automatizzabile la produzione di exploit efficaci. Basti pensare a quelli utilizzati nei
test di questo lavoro di tesi, grazie ad essi alcune delle più complesse operazioni risultano notevolmente semplificate, dando la possibilità di creare attacchi articolati e quasi completamente automatizzabili anche per utenti meno esperti.\\
La \textbf{Return Oriented Programming} e tutte le altre tecniche derivanti da essa, rappresentano quindi una minaccia tanto potente quanto pericolosa, costituendo di fatto una problematica sia presente sia futura.\\
Tutti gli attacchi descritti nelle sezioni precedenti dimostrano come questa tecnica possa risultare estremamente versatile ed adattabile ad ogni tipologia di vulnerabilità presentata. Tuttavia, anch'essa possiede diverse limitazioni
che la possono rendere evitabile. Nello specifico è possibile notare come per alcuni degli attacchi fosse necessaria la presenza di determinate condizioni per poter eseguire con successo un attacco completo. Nella realtà dunque, tale tecnica 
non sempre sarà utilizzabile poiché tali condizioni non saranno sempre verificabili.\\
In conclusione, la tecnica d'attacco informatico \textbf{ROP}, nonostante la sua apparente semplicità può essere ancora considerata uno strumento pericoloso se utilizzata da utenti malintenzionati. Essa potrebbe portare conseguenze rilevanti soprattutto in un periodo storico particolarmente caratterizzato dallo sviluppo tecnologico come quello attuale.